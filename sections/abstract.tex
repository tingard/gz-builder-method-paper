% !TEX root = ../main.tex
\documentclass[../main.tex]{subfiles}
\begin{document}
Multi-component modelling of galaxies is a valuable tool in the quantitative understanding of galaxy evolution, however it is plagued by issues with convergence, model selection and parameter degeneracies. These issues either limit it to simple models for large samples, or complex models in very small samples  (a dilemma we summarize as a choice between “quality or quantity”). This paper presents a novel framework, built inside the Zooniverse citizen science platform, which enables volunteers to help crowdsource the creation of multiple component photometric models of galaxies from FITS images. We test if this method can help solve the quandry over choosing ``quality" or ``quantity" for complex galaxy image modelling.

We have run the method (including a final algorithmic optimization from the best crowd-sourced solution) on a sample of 198 galaxies from the Sloan Digital Sky Survey. We examine the robustness of this new method to variation in number and population of citizen scientists, as well as compare it to automated fitting pipelines. We demonstrate that it is possible to consistently recover accurate models which show good agreement with, or improve on previous models in the literature. We demonstrate that using citizen science to make selections on number of model parameters to include and their rough optimal values is a promising technique for modeling the images of complex galaxies. We release our catalogue of models to the community.
\end{document}

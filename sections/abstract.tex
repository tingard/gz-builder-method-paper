% !TEX root = ../main.tex
\documentclass[../main.tex]{subfiles}
\begin{document}
This paper presents a novel interface, built inside the Zooniverse citizen science platform, which enables volunteers to help to create detailed photometric models of galaxies from SDSS images. The value of citizen science to obtain classifications of galaxy morphology on a large scale has been widely demonstrated in the literature, however such morphologies can lack the quantitative power obtained through the fitting of photometric profiles. Multi-component modelling of complex galaxies is plagued by issues with convergence, model selection and parameter degeneracies, which we demonstrate can be adressed using citizen science. We examine the consistency of this new method with changes in number and population of citizen scientists and compare it to more traditional automated fitting pipelines \comment{Chris: how does it get on?}. These results will be used in future work to investigate spiral arm formation mechanisms and we release our catalogue of models to the community.
\end{document}

% !TEX root = ../main.tex
\documentclass[../main.tex]{subfiles}
\begin{document}
\section{Technical overview of the software}

\section{Mathematical description of the poly-line separation measure}
\label{appendix:clusteringMaths}

First, define a poly-line containing $n$ pairs of x- and y-values (vertices) as

\begin{equation}
A: \{i \in \mathbb{N};\;i<n\} \longrightarrow \mathbb{R}
^2\end{equation}

We also define a function, $t$, which calculates how far a point $\vec{p}$ is along the line between two other points ($\vec{v}$ and $\vec{w}$):

\begin{equation}
t(\vec{p},\,\vec{v},\,\vec{w}) \equiv \frac{(\vec{p} - \vec{v})\cdot(\vec{v} - \vec{w})}{|\vec{w} - \vec{v}|^2}.
\end{equation}

The minimum distance from $\vec{p}$ to the line segment between $\vec{v}$ and $\vec{w}$ is given by

\begin{equation}
d(\vec{p},\,\vec{v},\,\vec{w}) = \|\left(\vec{v} + \mathrm{min}(\mathrm{max}(t(\vec{p},\,\vec{v},\,\vec{w}),\, 0),\, 1)\;(\vec{w} - \vec{v})\right) - \vec{p}\|
\end{equation}

We then define a ``squared distance'' from the poly-line $A$ (containing $n$ vertices) to the poly-line $B$ (containing $m$ vertices):

\begin{equation}
D(A,\,B) \equiv \sum_{i = 0}^{n} \mathrm{min}\{j \in \mathbb{N}_0,\, j < m;\; d(A_i,\, B_j,\, B_{j+1})^2\}.
\end{equation}

The choice to square the distances and penalize large deviations from other lines was a data-driven choice to improve the results of clustering.

Finally, we define our separation measure between two drawn poly-lines as

\begin{equation}
distance(A,\,B) \equiv D(A,\,B) + D(B,\,A).
\end{equation}

\comment{Some figures showing how this distance is calculed would be useful}

\section{Example resulting splines from Galaxy Builder}
\label{appendix:exampleSplines}

\begin{figure}
  \includegraphics[width=8cm]{images/montage-subject-21096891.jpg}
  \caption{An example of the resulting spiral classifications for a galaxy in the Galaxy Builder sample}
\end{figure}
\end{document}

\section{Model Rendering Profile}

Standard S\`ersic \comment{citation}:
\begin{equation}
  I(R) = I_e \exp\left(-b_n\left[\left(\frac{R}{R_e}\right)^{1/n} - 1\right]\right)
\end{equation}

As implemented in Galaxy Builder (for a galaxy with axis ratio $E$, drawn isophotal major axis $r_e$, boxyness $c$ and S\`ersic index $n$):

\begin{equation}
  I(R) = \frac{I_0}{2} \exp\left(-b_n\left[D^{1/n} - 1\right]\right),
\end{equation}

\noindent
where

\begin{equation}
  \begin{aligned}
    D &= 3\sqrt{\left(\frac{E}{r_e}\right)^c|x' - \mu_x|^c + \left(\frac{1}{r_e}\right)^c|y'-\mu_y|^c} \\
      &= \frac{3}{r_e^{c/2}}\sqrt{E^c|x' - \mu_x|^c + |y'-\mu_y|^c}
  \end{aligned}
\end{equation}

\noindent
and

\begin{equation}
  \begin{pmatrix}x'\\y'\end{pmatrix} =
    \begin{pmatrix}\cos\phi & -\sin\phi\\\sin\phi & \cos\phi\end{pmatrix}\begin{pmatrix}x\\y\end{pmatrix}
      + \vec{\mu}
      - \begin{pmatrix}\cos\phi & -\sin\phi\\\sin\phi & \cos\phi\end{pmatrix} \vec{\mu}.
\end{equation}

\noindent

Therefore the effective radius is given by

\begin{equation}
  R_e = \frac{r_e^{c/2}}{3}
\end{equation}

% !TEX root = ../aas_submission.tex
\documentclass[../main.tex]{subfiles}
\begin{document}

\section{Model Fitting}
\label{sec:appendix_model_fitting}
Assume Normal priors on component parameters determined from clustering ($\mu_x$, $\mu_y$, $q$, $Re$), with the spread given by the spread in the clustered values. We therefore have that our final log-likelihood (to be maximised) is the sum of the gaussian log-likelihood of the residuals given the pixel uncertainty and the gaussian log-likelihood of the variation in parameters, given their uncertainty.

The model being rendered is the PSF-convolved sum of the separate components and outputs an ($N_x$, $N_y$) image. The disc, bulge and bar are variations on the boxy S\'ersic profile:

\begin{equation}
I_\mathrm{sersic}(\vec{P}) = I_e \exp\left\{-b_n\left[\left(\frac{r\,(\vec{P})}{R_e}\right)^{1/n} - 1\right]\right\}
\end{equation}

where

\begin{equation}
r\,(\vec{P}) = \left|\begin{pmatrix}
\frac{1}{q} & 0 \\
0 & 1
\end{pmatrix}\begin{pmatrix}
\cos\psi & -\sin\psi\\
\sin\psi & \cos\psi
\end{pmatrix}\left(\vec\mu - \vec{P}\right)\right|_{\ c}.
\end{equation}

The disc is resticted to $n=1; c=2$, bulge to $n\in(0.5, 6); c=2$ and bar to $n\in(0.5, 6); c\in(0.5, 6)$.

The S\'ersic components are actually rendered at 5x the image resolution, and downsampled using the mean pixel brightness. This is a widely used method of approximating the true pixel value, which is an integration over the area of sky inside the pixel. I.e. for a pixel of size $(\delta_x, \delta_y)$

\begin{equation}
I_\mathrm{pix}(\vec{P}) = \frac{1}{\delta_x \delta_y}\int_{-\delta_y/2}^{\delta_y/2}\int_{-\delta_x/2}^{\delta_x/2}\mathrm{d}x\mathrm{d}y\ I_\mathrm{sersic}\left(\vec{P} + \begin{pmatrix}
\delta_x \\
\delta_y \\
\end{pmatrix}\right).
\end{equation}

Spiral arms were restricted to be logarithmic with respect to the inclined, rotated disc. They were rendered in a similar manner to the online interface; using the nearest distance from a pixel to a calculated logarithmic spiral.

An inclined, rotated log spiral requires parameters brightness $I_s$, spread $s$, minimum and maximum $\theta$ ($a$ and $b$), an amplitude $A$, pitch angle $\phi$, position $\vec\mu$, position angle $\psi$ and axis ratio $q$, where $\vec\mu$, $\psi$ and $q$ are inherited from the disc component.

The distance from a pixel to a logarithmic spiral is given by
\begin{equation}
  D_\mathrm{s}(\vec{P}) = \min_{\theta\in[a, b]}\left|\left|\vec{P} - \vec\mu - Ae^{\theta\tan\phi}\begin{pmatrix}
       \cos\psi & \sin\psi\\
       -\sin\psi & \cos\psi
       \end{pmatrix}
       \begin{pmatrix}
       q\cos\theta \\
       \sin\theta \\
       \end{pmatrix}
       \right|\right|^{\ 2}.
\end{equation}

In practise the spiral distance was approximated using the distance to a poly-line with 200 vertices, as solving the above minimization for each pixel at each fitting step is computationally intractable. We also adjust $A$, $a$ and $b$ to account for the rotation of the disc component from its starting value, in order to prevent spirals inadvertendly moving far from starting locations for face-on discs (which have poorly constrained position angles). These adjustments are

\begin{equation}
\begin{aligned}
  A' &= Ae^{\Delta\psi\tan\phi},\\
  a' &= a - \Delta\psi,\\
  b' &= b - \Delta\psi.
\end{aligned}
\end{equation}

The pixel brightness is then calculated as

\begin{equation}
I_\mathrm{spiral}(\vec{P}) = I_\mathrm{disc}(\vec{P}) \times I_s\exp\left(\frac{-D_\mathrm{s}(\vec{P})}{2s^2}\right).
\end{equation}

For the fit, we parametrize disc $I_e$ as the S\'ersic total luminosity, given by

\begin{equation}
L_\mathrm{tot} = I_eR_e^2\ 2\pi n\frac{e^{b_n}}{(b_n)^{2n}}\Gamma(2n).
\end{equation}

Bulge (bar) $I_e$ is reparametrized as ``bulge (bar) fraction'', which we define as

\begin{equation}
F_\mathrm{bulge} = \frac{L_\mathrm{bulge}}{L_\mathrm{disc} + L_\mathrm{bulge}},
\end{equation}

and is limited to be between 0 and 1. Disc luminosity is allowed to take any value greater than or equal to zero.

Similarly, bulge and bar effective radius are reparametrized as their scale relative to the disc ($R_e = R_e / R_{e,\,\mathrm{disc}}$). Bulge and bar are also restricted to have the same poition.

\end{document}

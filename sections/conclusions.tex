% !TEX root = ../aas_submission.tex
\documentclass[../main.tex]{subfiles}
\begin{document}
\label{sec:conclusions}
In this paper we present a novel method for modelling of galaxy images, \textit{Galaxy Builder}, which was conceived with the goal of solving the ``quality or quantity" dilema facing galaxy image modelling, which, despite advances in computation, still typically requires significant human interaction to achieve quality fits. We release our catalogue of models to the community, and in future work we use this sample to investigate spiral arm formation mechanisms.

\textit{Galaxy Builder} leverages the power of crowd sourcing for the hardest to automate parts of image fitting, namely determining the appropriate number of model components to include, and finding regions of parameter space close to the global optima.

The use of a small sample of synthetic images to calibrate and test our model clustering and fitting code has demonstrated our ability to recover galaxy morphology in the majority of cases. For example, our spiral arm fitting recovered spiral pitch angles to within 9 deg. This set of 9 synthetic images revealed a systematic tendency for volunteers to incorporate more bulges and fewer bars than necessary for photometric models of strongly barred spirals. Future work might implement an improved clustering algorithm (and improved user interface) to address the failures of bar model clustering we observed in a small fraction of galaxies.

Our fitting method is still subject to the limitations of gradient-descent based optimization, and will be trapped by local minima. This is highly impactful for parameters for which clustering provides no prior information (bulge and bar S\'ersic $n$, bar boxyness), future work should attempt to use a more robust algorithm (such as Basin-Hopping, \citealt{1998cond.mat..3344W}).

We have demonstrated our ability to obtain physically motivated models with comparable reduced chi-squared values (between 1 and 5) to results in the literature. We obtain errors on parameters where possible through the sample standard deviation of component clusters, which is less likely to be an under-estimate than approximations using the local curvature of the Likelihood-space.

We compare these new models to existing results in the literature. We find good agreement where the models or parameters are comparable, and comment on instances where \textit{Galaxy Builder} should provide superior models because of the more realistic modelling of the components.

We were able to obtain aggregate models for 296 images with an average rate of one galaxy per day, and fit photometric models for 294 images. User experience and task simplification will need to be considered if significantly larger numbers of these models are to be obtained. At the time of writing and to the best of our knowledge, the number of photometric models obtained here is still significantly larger than the largest sample obtained through purely computational photometric fitting of disc, bulge, bar and spiral arms in galaxies (10 galaxies, \citealt{Gao2017:1709.00746v1}, who also included rings, disc-breaks and further components).

Any citizen science project is only as good as the volunteers who generously donate their time to it. We were incredibly fortunate to be able to make use of the wonderful pool of volunteers built by the Zooniverse, who in some cases contributed hundreds of detailed galaxy classifications to this project. We are optimistic about the potential of projects like \textit{Galaxy Builder} to dramatically increase the ability of researchers to perform complex, labour-intensive modelling of galaxy photometry, leveraging the power of the crowd to perform the complex tasks best suited to humans, and computer algorithms for the final optimization.
\end{document}

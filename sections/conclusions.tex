% !TEX root = ../main.tex
\documentclass[../main.tex]{subfiles}
\begin{document}
\label{sec:conclusions}
This paper has explored how citizen science could impact the problem of photometric modelling of spiral galaxies using complex models. We detailed a novel project inside the popular and successful Zooniverse citizen science platform that enables galaxy photometric decomposition in near-realtime.

We have demonstrated that it is possible to obtain models from volunteer classifications, using both the best individual classification and an aggregate model from clustering, which show good agreement with existing results in the literature.

The complexity of the tasks involved in this project coupled with the difficulties in clustering and aggregating data suggest that this is one problem space the power of the crowd hasn't quite overcome yet. Despite this, we remain of the opinion that better user experience design \& in-browser computer optimization will help improve the quality of classifications and thus that of the recovered models.

We are optimistic about the potential of projects like Galaxy Builder to dramatically increase the ability of researchers to perform complex, labour-intensive modelling of galaxy photometry.

\end{document}

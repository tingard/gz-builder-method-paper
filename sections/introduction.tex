% !TEX root = ../aas_submission.tex
\documentclass[../main.tex]{subfiles}
\begin{document}
\label{sec:introduction}

Disc galaxies are complex objects, containing many different components, including a disc, disc phenomena (i.e. spiral arms, bars and rings) and central, more spheroidal structures (bulges, bars). Decomposing disc galaxies into their component structures has become an important tool for extragalactic astronomers seeking to understand the formation and evolution of the galaxy population (e.g. \citealt{Simard2002:astro-ph/0205025v2}, \citealt{2011ApJS..196...11S}, \citealt{2012MNRAS.421.2277L}, \citealt{2017MNRAS.469.3363K}, \citealt{megamorph-paper}, \citealt{2011MNRAS.415.3308G}, \citealt{Mendez-Abreu2016:1610.05324v1}, \citealt{Park2006:astro-ph/0611610v2}, \citealt{2015ApJS..219....4S}).

These fully quantitative methods allow researchers to obtain structural parameters of galaxy sub-components, which has use in a variety of astrophysical and cosmological research. For example, the stellar mass found in discs and bulges places strong constraints on the galaxy merger tree from $\Lambda\mathrm{CDM}$ N-body simulations (\citealt{Hopkins2010:1004.2708v3}, \citealt{2018MNRAS.475.5133R}, \citealt{2009MNRAS.396.1972P}); the strength of a galaxy's classical bulge is thought to be tied to the strength of a merger event in its past (\citealt{Kormendy2010:1009.3015v1}, \citealt{2005ApJ...622L...9S}); different spiral arm formation theories vary in their predictions of spiral morphology (\citealt{Dobbs2014:1407.5062v1}, \citealt{Pour-Imani2016:1608.00969v1}, \citealt{2017MNRAS.472.2263H}).

The usefulness of obtaining parametric models of a galaxy has motivated the creation of many image modelling and fitting suites, including \textsc{Gim2d} \citep{gim2d-paper}, \textsc{Galfit} \citep{galfit-paper}, \textsc{MegaMorph} \citep{megamorph-paper} and \textsc{Profit} \citep{profit-paper} to name a few. Using these tools, researchers have built large catalogues of model fits to galaxies. Perhaps most notably \citet{2011ApJS..196...11S} performed two-dimensional, Point-Spread-Function (PSF) convolved, two-component (bulge + disc) decomposition of 1,123,718 galaxies from the Legacy imaging of the Sloan Digital Sky Survey (hereafter SDSS) Data Release 7 \citep{SDSSDR7}. Other large catalogues of photometric fits exist: \cite{2011MNRAS.415.3308G} made use of parametric multi-band light distribution modelling to model stellar bars in 300 galaxies, \cite{Mendez-Abreu2016:1610.05324v1} made use of a human-supervised approach to perform multi-component decomposition of 404 galaxies from the CALIFA survey \citep{Sanchez2011:1111.0962v2}.

However, despite the usefulness of this technique and the presence of analytic profiles and methods for modelling more complex galaxy sub-components, relatively few studies have attempted to perform large-scale (1000s of galaxies) parametric decomposition of galaxies using more complicated models than that of \citet{2011ApJS..196...11S}. Not properly taking into account these ``secondary'' morphological features (such as a bar, ring and spiral arms) can impact detailed measurements of a galaxy's bulge \citep{Gao2017:1709.00746v1}. Proper decomposition of secondary morphological features allows investigation into mechanisms behind the secular evolution of galaxies (\citealt{2018MNRAS.473.4731K}, \citealt{2018ApJ...862..100G}, \citealt{2015MNRAS.453.3729H}) and exploration of environmental effects on morphology, such as offset bars \citep{2017MNRAS.469.3363K}.

A prominent issue when performing these detailed decompositions is the tendency for fitting functions to converge on unphysical results when not properly guided or constrained, for instance a S\'ersic bulge swapping places with an exponential disc component (as observed by \citealt{2018MNRAS.473.4731K}). It is also the case that often, without near-optimal starting points, detailed model fits will fail to converge at all \citep{2016MNRAS.462.1470L}.

Compounding this, uncertainties reported by many software fitting packages (i.e. \textsc{Galfit} and \textsc{MegaMorph} from the above list) are lower estimates on the real uncertainty, due to secondary sources not being modelled, flat-fielding errors and incorrect models \citep{2010AJ....139.2097P}. Other packages such as \textsc{Gim2d} and \textsc{Profit} attempt to fully model posterior distributions and so produce more representative uncertainties, however this comes with a larger computational cost and configuration complexity. These uncertainties are all measures of the likelihood space, and are also only truly valid if the model used is correct.

Another problem which needs to be addressed is whether a component should be present in the model at all. An automated fit will generally attempt to add as many components as possible to produce the closest-matching model. Many studies therefore need to select the most appropriate model by visual inspection of the resulting residuals or recovered parameters. For example, both \citet{Vika2014:1408.4070v1} and \citet{2018MNRAS.473.4731K} inspected the resulting model and residual images for all of their parametric fits (163 and 5,282 respectively) to ensure physical results with the correct components present. The end result of most of these problems is that researchers will have to invest time to individually check many of their fits to ensure they have converged on a physical model. In the era of large sky surveys such as the SDSS \citep{SDSSDR7}, which in total imaged over 50 million galaxies, the time required to do this becomes unsustainable and introduces concerns over human error if done by only a single, or small number of individuals.

A demonstrably successful solution to the similar problem of galaxy classification in the era of large surveys, was to find a new source of person-power: \cite{Lintott2008:0804.4483v1} invited large numbers of people to classify SDSS-images of galaxies over the internet in the Galaxy Zoo project. The resulting classifications (a mean of 38 per galaxy) were then weighted and averaged to create a morphological catalogue of 893,212 galaxies. This hugely successful project, including its subsequent iterations and expansions (i.e. \citealt{Willett2013:1308.3496v2}, \citealt{Hart2016:1607.01019v1}, \citealt{2017MNRAS.464.4176W}, \citealt{2017MNRAS.464.4420S}), has produced a large catalogue of detailed morphological classifications which are in good agreement with other studies, and have been used in a wide variety of studies of the local galaxy population (see \citealt{2019arXiv191008177M} for a recent review).

In this paper we explore an analogous solution to that \citet{Lintott2008:0804.4483v1} brought to galaxy classification for the issues faced by galaxy parametric modelling, inside the ecosystem that Galaxy Zoo set in motion (namely {\it The Zooniverse}\footnote{\url{https://www.zooniverse.org}}). We leverage citizen scientists to pick model components and perform model optimization in an online, web-browser environment\footnote{\url{https://www.zooniverse.org/projects/tingard/galaxy-builder}}. We describe our method in Section \ref{sec:method}, including details of the images and ancilliary data from SDSS as well as the strategy used to obtain scientifically useful models from volunteer classifications. We provide consistency checks within our infrastructure and to other methods in Section \ref{sec:results}.

Where necessary, we make use of $H_0 = 70\ \text{km}\ \text{s}^{-1}\ \text{Mpc}^{-1}$.

\end{document}

% !TEX root = ../main.tex
\documentclass[../main.tex]{subfiles}
\begin{document}
\label{sec:introduction}

Disc galaxies are complex objects, containing many different components, including a disc and disc phenomena (i.e. spiral arms, bars and rings), as well as central more spheroidal structures (bulges, nuclear bulges). Decomposing disc galaxies into their component structures has become an important tool for extragalactic astronomers seeking to understand the formation and evolution of the galaxy population (e.g. \citealt{Simard2002:astro-ph/0205025v2}, \citealt{Simard2011:1107.1518v1} \citealt{2012MNRAS.421.2277L}, \citealt{2017MNRAS.469.3363K}, \citealt{megamorph-paper}, \citealt{Gadotti2010:1003.1719v2} \citealt{Mendez-Abreu2016:1610.05324v1}, \citealt{Park2006:astro-ph/0611610v2}).



%%% SAVE FOR THESIS
%\subsection{Galaxy Morphology}
%label{subsec:galaxy-morphology}
%One of the cornerstones of all areas of empirical science over the past few centuries has been the sifting-through and sorting of objects of interest into types; this has been true of the field of galaxy classification since the mid-1930s, when Hubble began developing his tuning-fork model of the galaxies observed in photographic images of the sky. \citep{Hubble1936}.

%It is commonly accepted that this tuning fork, even with its subsequent expansions (for example that used in \citealt{Sandage1961} and \citealt{deVaucouleurs1991}), is too simplistic and subjective a measure for systems as complex as galaxies. One of the tuning fork's greatest strengths is that it provides a framework for basic ideas, including the notion that galaxies are comprised of distinct components such as discs, bulges, spiral arms and bars. Recently, it appears that expert classification of galaxies has shifted away from Hubble's original spiral tightness classification for spiral galaxies and towards bulge size, which appears not to be correlated with spiral tightness \citep{Masters2019:1904.11436v1}.

%Researchers have begun to explore alternate methods of classification, including the use of rotation curves and internal dynamics (\citealt{2011MNRAS.413..813C}, \citealt{2017MNRAS.469.2539K}, \citealt{Fall2018:1812.06144v1}).

%One major problem encountered in morphological classification is that it is difficult to avoid subjectjivity in a classification scheme: \citet{Naim1995:astro-ph/9502078v1} noted that between their sample of six experts, a galaxy would only receive a 50\% consensus on Hubble type (more than three identical, independent classifications from the experts) 55\% of the time, with all experts agreeing independently on a classification for only 8 of their sample of 354 galaxies.

%Grouping galaxies based on their morphology also results in the clustering of other physical parameters such as star formation rate or gas fraction (\citealt{RobertsHaynes1994} provides a detailed discussion of many such correlations \comment{A more recent citation would be better}). It also enables the selection of morphology-based sample sets from which physical processes and characteristics can be probed.

%Until the late 20\textsuperscript{th} Century, it was possible for small teams to band together and compile catalogues of classifications of many of the well-observed galaxies at the time (i.e The Third Reference Catalog of Bright Galaxies, \citealt{deVaucouleurs1991}, containing 18,000 classifications, or the ESO catalog of galaxies, \citealt{1989Msngr..56...31L}, containing 15,000 classifications). However, with the beginning of the era of large sky surveys such as the Sloan Digital Sky Survey (hereafter SDSS, \citealt{2017AJ....154...28B}, \citealt{SDSSDR7}, over 50 million galaxies), the time required for classifying galaxies grew to unsustainable levels for most research teams. See \citet{2010yCat..21860427N}, \citet{2007ApJS..173..512S} for the largest expert efforts. \citet{Naim1995:astro-ph/9502078v1} was one of the first to discuss the need to move to automated photometric classification, and investigated possible methods of automation, including the use of an Artificial Neural network.

%One approach to solving the problem of large-scale classification was to identify proxies for morphology, such as colour or concentration index (for example, the ``Zest'' method of  \citet{Scarlata2007:astro-ph/0701746v2}). These methods could then be easily rolled-out to the required scales. However, the use of proxies introduces some unknown bias in the resulting classifications (i.e. not all spiral galaxies have blue outer regions). Different sample sets created using different morphological proxies will differ in some difficult-to-quantify manner, due to the underlying nature of the galaxies being classified. The usage of catalogues created using some form of morphological proxy may also be statistically unsound for particular science cases; for instance studying star formation rates using a catalogue compiled using optical colours will obviously produce a biased result.

%This trend towards use of colour, or automated proxies for morphology (such as bulge size), has also resulted in a subtle shift in what we mean by certain morphological terms (like late- or early-type), so care must be taken comparing with older morphological based classifications (see \citealt{Masters2019:1904.11436v1} for a longer discussion of this point).

%\subsection{Light Distribution Modelling}

%Another way to characterise an image of a galaxy is to make use of analytic functions to model the components of that galaxy.

These fully quantitative methods allow researchers to obtain structural parameters of galaxy sub-components, which are useful in a variety of astrophysical and cosmological research. For example, the stellar mass found in discs and bulges places strong constraints on the galaxy merger tree from $\Lambda\mathrm{CDM}$ N-body simulations \citep{Hopkins2010:1004.2708v3}; the strength of a galaxy's classical bulge is thought to be tied to the strength of a merger event in its past \citep{Kormendy2010:1009.3015v1}; different spiral arm formation theories slightly vary in their predictions of spiral morphology (\citealt{Dobbs2014:1407.5062v1}, \citealt{Pour-Imani2016:1608.00969v1} \citealt{2017MNRAS.472.2263H}).

The usefulness of obtaining parametric models of a galaxy has motivated the creation of many image modelling and fitting suites, including \textsc{Gim2d} \citep{gim2d-paper}, \textsc{Galfit} \citep{galfit-paper}, \textsc{MegaMorph} \citep{megamorph-paper} and \textsc{Profit} \citep{profit-paper} to name a few. Using these tools, researchers have built large catalogues of model fits to galaxies. Perhaps most notably \citet{Simard2011:1107.1518v1} performed two-dimensional, Point-Spread-Function (PSF) convolved, two-component (bulge + disc) decomposition of 1,123,718 galaxies from the Legacy area of the SDSS DR7. Other large catalogues of photometric fits exist: \cite{Gadotti2010:1003.1719v2} made use of parametric multi-band light distribution modelling to model stellar bars in 300 galaxies, \cite{Mendez-Abreu2016:1610.05324v1} made use of a human-supervised approach to perform multi-component decomposition of 404 galaxies from the CALIFA survey \citep{Sanchez2011:1111.0962v2}.

However, despite the usefulness of this technique and the presence of analytic profiles and methods for modelling more complex galaxy sub-components, relatively few studies have attempted to perform large-scale parametric decomposition of galaxies using more complicated models than that of \citet{Simard2011:1107.1518v1}. Not properly taking into account these ``secondary'' morphological features (such as a bar, ring and spiral arms) can impact detailed measurements of a galaxy's bulge \citep{Gao2017:1709.00746v1}. Proper decomposition of secondary morphological features allows investigation into mechanisms behind the secular evolution of galaxies (\citealt{2018MNRAS.473.4731K}, \citealt{2018ApJ...862..100G}) and exploration of environmental effects on morphology, such as offset bars \citep{2017MNRAS.469.3363K}.

A prominent issue when performing these detailed decompositions is the tendency for fitting functions to converge on unphysical results simply when not properly guided or constrained, for instance a S\'ersic bulge swapping places with an exponential disc component. The extra degree of freedom of the S\'ersic profile allows a more desirable residual \citep{2017MNRAS.469.3363K}. It is also the case that often, without near-optimal starting points, detailed model fits will fail to converge at all.

Another problem which needs to be addressed is whether a component should be present in the model at all. An automated fit will generally attempt to add as many components as possible to produce the closest-matching model. Many studies therefore need to select the most appropriate model by visual inspection of the resulting residuals or recovered parameters. For example, both \citet{Vika2014:1408.4070v1} and \citet{2018MNRAS.473.4731K} inspected the resulting model and residual images for all of their parametric fits (163 and 5,282 respectively) to ensure physical results with the correct components present. The end result of most of these problems is that researchers will have to invest time to individually check many of their fits to ensure they have converged on a physical model. In the era of large sky surveys such as the Sloan Digital Sky Survey (hereafter SDSS, \citealt{2017AJ....154...28B}, \citealt{SDSSDR7}, which in total imaged over 50 million galaxies), the time required to do this becomes unsustainable and introduces concerns over human error if done by only a single, or small number of individuals.
 %, meaning we are faced with a similar problem of scale discussed in Section \ref{subsec:galaxy-morphology}.

%\subsection{Citizen Science/Crowd Sourcing}

A demonstrably successful solution to the similar problem of galaxy classification in the era of large surveys, was to find a new source of person-power: \cite{Lintott2008:0804.4483v1} invited large numbers of people to classify SDSS-images of galaxies over the internet in the Galaxy Zoo project. The resulting classifications (a mean of 38 per galaxy) were then weighted and averaged to create a morphological catalogue of 893,212 galaxies. This hugely successful project, including its subsequent iterations and expansions (i.e. \citealt{Willett2013:1308.3496v2}, \citealt{2017MNRAS.464.4176W}, \citealt{2017MNRAS.464.4420S}, \citealt{Hart2016:1607.01019v1}), has produced a large catalogue of detailed morphological classifications which are in good agreement with other studies, and have been used in a wide variety of studies of the local galaxy population
(see \citealt{2019AAS...23333201M} for a recent review).

In this paper we explore an analogous solution to that Galaxy Zoo brought to galaxy classification \citep{Lintott2008:0804.4483v1} for the issues faced by galaxy parametric modelling inside the ecosystem that Galaxy Zoo set in motion (namely {\it The Zooniverse}\footnote{\url{https://www.zooniverse.org}}; leveraging citizen scientists to pick model components and perform model optimization in an online, web-browser environment\footnote{\url{https://www.zooniverse.org/projects/tingard/galaxy-builder}}). We describe our method in Section \ref{sec:method}, including details of the images and ancilliary data from SDSS as well as the strategy used to obtain scientifically useful models from volunteer classifications. We provide consistency checks within our infrastructure and to other methods in Section \ref{sec:results}.

Where necessary, we make use of $H_0 = 70\ \text{km}\ \text{s}^{-1}\ \text{Mpc}^{-1}$.

\end{document}

% !TEX root = ../main.tex
\documentclass[../main.tex]{subfiles}
\begin{document}
\subsection{Galaxy Morphology}
\label{subsec:galaxy-morphology}
One of the cornerstones of science over the past few centuries (and before) has been the sifting-through and sorting into types of objects of interest; this has been true of the field of galaxy classification since the mid-1930s, when Hubble began applying his tuning-fork model to the galaxies observed in the sky. \citep{Hubble1936}.

It is commonly accepted that this tuning fork, even with its subsequent expansions (for example that used in \citealt{Sandage1961} and \citealt{deVaucouleurs1991}), is too simplistic and subjective a measure for systems as complex as galaxies. One of the tuning fork's greatest strengths is that it provides an important grounding of basic ideas, including the notion that galaxies are comprised of distinct components such as discs, bulges, spiral arms and bars. Researchers have begun to explore alternate methods of classification, including the use of rotation curves and internal dynamics (\citealt{2011MNRAS.413..813C}, \citealt{2017MNRAS.469.2539K}, \citealt{Fall2018:1812.06144v1}). One major advantage of this switch in methodology is the removal of subjective opinion from a catalog of classification: \citet{Naim1995:astro-ph/9502078v1} noted that between their sample of six experts, a galaxy would only receive a 50\% consensus on Hubble type (more than three identical, independent classifications from the experts) 55\% of the time, with all experts agreeing independently on a classification for only 8 of their sample of 354 galaxies.

Grouping galaxies based on their morphology also results in the clustering of other physical parameters such as star formation rate or gas fraction (\citealt{RobertsHaynes1994} provides a detailed discussion of many such correlations). It also enables the selection of morphology-based sample sets from which physical processes and characteristics can be probed. \comment{Think we need to cite some non-GZ people saying this is too simple.}

Until the late 20\textsuperscript{th} Century, it was possible for small teams to band together and compile catalogues of classifications of many of the well-observed galaxies at the time (i.e The Third Reference Catalog of Bright Galaxies, \citealt{deVaucouleurs1991}, containing 18,000 classifications, or the ESO catalog of galaxies, \citealt{1989Msngr..56...31L}, containing 15,000 classifications). However, with the beginning of the era of large sky surveys such as the Sloan Digital Sky Survey, hereafter SDSS (\citealt{SDSSDR7}, over 50 million galaxies), the time-cost required for classifying galaxies grew to unsustainable levels for research teams. \citet{Naim1995:astro-ph/9502078v1} was one of the first to discuss the need to move to automated photometric classification, and investigated possible methods of automation, including the use of an Artificial Neural network.

One approach to solving the problem of large-scale classification was to identify proxies for morphology, such as colour or concentration index (for example, \citet{Scarlata2007:astro-ph/0701746v2}). These methods could then be easily rolled-out to the required scales. However, the use of proxies introduces some unknown bias in the resulting classifications (i.e. not all spiral galaxies have blue outer regions). Different sample sets created using different morphological proxies would in some difficult-to-quantify manner, due to the underlying nature of the galaxies being classified. The usage of catalogues created using some form of morphological proxy may also be statistically unsound for particular science cases; for instance studying star formation rates using a catalogue compiled using optical colours would produce a biased result.

\subsubsection{Citizen Science}
A promising solution to large-scale classification was to find a new source of person-power: \cite{Lintott2008:0804.4483v1} invited large numbers of people to classify SDSS-images of galaxies over the internet in the Galaxy Zoo project. The resulting classifications (roughly 38 per galaxy) were then weighted and averaged to create a morphological catalogue of 893,212 galaxies.

This hugely successful project, including its subsequent iterations and data improvement methodologies (i.e. \citealt{Hart2016:1607.01019v1}), has produced a large catalogue of detailed morphological classifications which are in good agreement with other studies (\citealt{Willett2013:1308.3496v2}, \citealt{Simmons2014:1409.1214v1}, \citealt{Willett2016:1610.03068v2}).

The questions presented to Galaxy Zoo volunteers have evolved over time. Originally asking if a galaxy was one of: smooth, a Z-wise spiral, an S-wise spiral, an edge-on spiral or a star or unknown or a merger event; the decision tree has now expanded to a more complex array of possibilities to better encompass the variety of galaxy morphologies.

However, while it is possible to draw quantitative measures from these classifications using vote fractions, they are predominantly descriptors of the visible structure present in the galaxy and don't provide a quantitative measurement of the relative size of different components, or allow the light from each to be isolated for further analysis.

\subsection{Light Distribution Modelling}

Another way to characterise an image of a galaxy is to make use of analytic functions to model the components of that galaxy. These fully quantitative methods allow researchers to obtain structural parameters of galaxy sub-components, which can be useful in a variety of astrophysical and cosmological manners: Stellar mass found in discs and bulges places strong constraints on the galaxy merger tree from $\Lambda\mathrm{CDM}$ N-body simulations \citep{Hopkins2010:1004.2708v3}; the strength of a galaxy's classical bulge is thought to be tied to the strength of a merger event in its past \citep{Kormendy2010:1009.3015v1}; different spiral arm formation theories slightly vary in their predictions of spiral morphology (\citealt{Dobbs2014:1407.5062v1}, \citealt{Pour-Imani2016:1608.00969v1} \citealt{Hart2017:1708.04628v1}).

The usefulness of obtaining parametric models of a galaxy has motivated the creation of many rendering and fitting suites, including \textsc{Gim2d} \citep{gim2d-paper}, \textsc{Galfit} \citep{galfit-paper}, \textsc{MegaMorph} \citep{megamorph-paper} and \textsc{Profit} \citep{profit-paper} to name a few. Using these tools, researchers have built large catalogues of model fits to galaxies. Perhaps most notably \citet{Simard2002:astro-ph/0205025v2} performed two-dimensional, Point-Spread-Function (PSF) convolved, two-component (bulge + disc) decomposition of 1,123,718 galaxies from the Legacy area of the SDSS DR7. Other catalogues of photometric fits exist: \cite{Gadotti2010:1003.1719v2} made use of parametric multi-band light distribution modelling to model stellar bars in 300 galaxies, \cite{Mendez-Abreu2016:1610.05324v1} made use of a human-supervised approach to perform multi-component decomposition of 404 galaxies from the CALIFA survey\citep{Sanchez2011:1111.0962v2}.

However, despite the usefulness of this technique and the presence of analytic profiles and methods for modelling more complex galaxy sub-components, relatively few studies have attempted to perform parametric decomposition of galaxies using more complicated models than that of \citet{Simard2002:astro-ph/0205025v2}. Not properly taking into account these ``secondary'' morphological features (such as a bar, ring and spiral arms) can impact detailed measurements of a galaxy's bulge, which is often seen as a record of its host physical processes and subsequent evolution.

A large part of the problem of performing these detailed decompositions is the tendency for fitting functions to wander away from physical results in the chase for the best possible residual. An example of this would be a S\'ersic bulge swapping places with an exponential disc component, as the extra degree of freedom of the S\'ersic profile will result in a more desirable residual \citep{Kruk2017:1705.00007v1}. It is also the case that often, without near-optimal starting points, detailed model fits will fail to converge at all.

A third problem which needs to be addressed is whether a component should be present in the model at all. An automated fit will generally attempt to add as many components as possible to produce the closest-matching model. Many studies therefore need to select the most appropriate model by visual inspection of the resulting residuals or recovered parameters. For example, \citet{Vika2014:1408.4070v1} eye-balled the resulting model and residual images for all 163 of their parametric fits to ensure consistency.

The end result of most of these problems is that researchers will have to eyeball many of their fits to ensure they have converged on a physical model and we are faced with a similar problem to that discussed in subsection \ref{subsec:galaxy-morphology}. This paper proposes an analogous solution to that introduced by \cite{Lintott2008:0804.4483v1}, inside the ecosystem that their research set in motion.

\comment{KLM: definitely ready for sharing. Minor comments above}

\end{document}

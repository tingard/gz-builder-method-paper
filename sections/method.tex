% !TEX root = ../aas_submission.tex
\documentclass[../main.tex]{subfiles}
\begin{document}
\label{sec:method}

\subsection{The \textit{Galaxy Builder} Zooniverse project}

\comment{CL: I think this could be shortened - it’s a bit chatty, and I’m not sure it adds much. I’d replace with just the first paragraph and then a line about Mergers and a note that we didn’t use Java.}

\textit{Galaxy Builder} is a citizen-science project built on the Zooniverse web platform. It asks volunteers to perform detailed photometric modelling of spiral galaxies (potentially including bulge, disc, bar and spiral arm components). A project of this kind, allowing volunteers to interact with and model data, had never been attempted inside the current Zooniverse web platform before, so this project involved designing and implementing a model rendering\footnote{We use the term rendering in a similar manner to that used for computer graphics: to calculate an image from a model or set of rules.} suite inside the existing Zooniverse front-end code-base. As with all citizen science solutions, we had to not only consider the accuracy of the resulting model, but also user experience and engagement in our design decisions.

The closest relative to this project within the Zooniverse ecosystem was the Galaxy Zoo: Mergers project \citep{Holincheck2016:1604.00435v1}. This project asked volunteers to help match the morphological properties of an image of merging galaxies to a plethora of restricted three-body simulations, in an attempt to identify the initial conditions that could result in the observed morphology. For part of the project, volunteers downloaded a Java applet, which would run restricted three-body simulations and generate output images. Volunteers could manipulate the model parameters used in the simulation, and vote on simulations which matched a given galaxy merger image or shared important tidal features. A new batch of simulations could then be run and an optimal solution converged on.

In many ways, this iterative workflow was very similar to that used in \textit{Galaxy Builder}: volunteers were asked to manipulate the parameters of a complex astrophysical model in order to identify the most likely solution, in a problem space that traditional computational modelling struggles to solve. However, \textit{Galaxy Builder} operates purely inside a web page and does not make use of secondary citizen science projects for model selection (such as the Galaxy Zoo: Mergers' \textit{merger wars} sub-project), instead using unsupervised clustering and computational optimization to identify final models.


\subsubsection{Project Timeline and Development}

The \textit{Galaxy Builder} project was built inside the Zooniverse's \citep{Simpson:2014:ZOW:2567948.2579215} \textsc{Panoptes-Front-End}\footnote{\url{http://github.com/zooniverse/Panoptes-Front-End}} codebase, using the \textsc{React.js}\footnote{\url{https://reactjs.org/}} framework, as well as WebGL\footnote{\url{https://www.khronos.org/webgl/}} to enable low-latency photometric galaxy model rendering. \textit{Galaxy Builder} entered a Zooniverse beta in late November 2017 and after some user experience improvements and code refactoring, the project was launched as an official Zooniverse project on the 24th of April 2018.

A major challenge during development of the project was finding the right balance between keeping a simple and intuitive interface and workflow while also allowing the freedom and versatility to properly model galaxies. It was also a significant challenge to develop a compelling and simple tutorial for what is one of the most complex projects attempted on the Zooniverse platform. Feedback from expert users was essential to this process as part of the typical beta trial process for Zooniverse projects\footnote{\url{https://help.zooniverse.org/best-practices/}}.


\subsubsection{The project interface}

The \textit{Galaxy Builder} project prompts volunteers to work through the step-by-step creation of a photometric model of a galaxy (described in detail in Section \ref{section:galaxy-model}). A screenshot of the interface can be seen in Figure \ref{fig:interfaceInProgress}, where a residual image is shown. The interface presents a volunteer with three views, which they can switch between at any time: a $r$-band cutout image of a spiral galaxy (see Section \ref{sec:data}), the galaxy model they have created so far, and the residual between their model and image (shown in blue and yellow).

\begin{figure}
  \plotone{images__method/interface_in_progress.jpg}
  \caption{The \textit{Galaxy Builder} interface. The residual image is being shown, and the volunteer is on the ``Disc'' task. The drawn disc component (yellow) is offset from the galaxy image (blue) to demonstrate the positive and negative residuals. Where the image equals the model the residual is black. The dots below the residual image allow the user to switch images. The icons to the right allow panning and zooming of the image (rotation was not functional for this project). The icons to the bottom right of the image allow colour inversion of the galaxy cutout, flagging of the image as inappropriate, inspection of galaxy metadata (i.e. sky position, link to SDSS SkyServer), ability to save the image as a favourite and to add to a Zooniverse ``collection''. The Score shown in the bottom left of the image is calculated using Equation \ref{eqn:gal_score} and is a rough goodness-of-fit measure.}
  \label{fig:interfaceInProgress}
\end{figure}


The workflow is designed so that volunteers slowly subtract increasing amounts of light from the galaxy, as is illustrated in Figure \ref{fig:residualsStepByStep}. A tutorial is available which contains a step-by-step guide to completing a classification. At each step volunteers are asked to first draw a simple isophote, and then make use of a series of sliders to adjust the parameters of the model component (see Section \ref{section:galaxy-model} for more information).

Volunteers are also guided by a ``score'', which is tied to the residuals and chosen to increase from zero to some arbitrary value depending on the galaxy; a less noisy and more easily modelled galaxy will have a higher maximum score. To map a residual image to a final score shown to volunteers we used

\begin{equation}
  \label{eqn:gal_score}
    S = 100 \exp\left(\frac{-A}{N}\sum_{i=0}^N\frac{\text{arcsinh}^2\left(\,|\text{y}_i - M_i|\ /\ 0.6\right)}{\text{arcsinh}\,0.6 }\right),
\end{equation}

where $N$ is the total number of pixels, $y$ is the cutout of the galaxy, normalized to a maximum value of 1 ($y = \text{cutout}/\text{max}(\text{cutout})$), $M$ is the model calculated by volunteers and $A=300$ is an arbitrary choice of scaling chosen based on a handful of test galaxies.

This score has the advantage of being easy (and fast) to generate from the residual image shown to volunteers (which was Arcsinh-scaled in a manner described by \citealt{Lupton2003:astro-ph/0312483v1}), however it is overly sensitive to small deviations of the model from the galaxy.

\begin{figure}
  \plotone{images/residualProgress.jpg}
  \caption{Figure demonstrating the desired result of each step of the modelling process, as seen from the residual image provided to volunteers. The top left panel shows the galaxy after only a disc component has been added: the top right contains a disc and a bulge; the bottom left has a disc, bulge and bar; the bottom right is the finished model with a disc, bulge, bar and spiral arms. The image shown is SDSS J104238.12+235706.8. This brightness and contrast of this image have been edited to improve visibility in print.}
  \label{fig:residualsStepByStep}
\end{figure}


\subsection{Sample Selection: Images and Ancillary Data}
\label{sec:data}

As a proposed solution to the problem of fitting multi-component and complex galaxies, \textit{Galaxy Builder} finds a niche with a sample of disc galaxies with spiral features. One such sample is the \textit{stellar mass-complete sample} in \citet{2017MNRAS.472.2263H}, which is a sample of face-on spiral galaxies with and without bars and selected to be complete across stellar masses $9.45 < \log(M_\star / M_\odot) < 11.05$. The test sample we use for the Galaxy Builder project was therefore selected from the \citet{2017MNRAS.472.2263H} sample of face-on spiral galaxies.

The morphological information required to select spiral galaxies came from the public data release of Galaxy Zoo 2 (\citealt{Willett2013:1308.3496v2}, hereafter GZ2). Each response to a GZ2 morphology question is allocated a $p$ value ranging from 0 to 1, where 0 indicates no volunteers responded positively to that question and 1 indicates all volunteers who classified that galaxy responded positively (i.e. $p_\text{bar} = 0.5$ would indicate $50\%$ of volunteers said a bar was present in a galaxy). Photometric measurements used for selection came from the NASA-Sloan Atlas (\citealt{2011AJ....142...31B}, hereafter NSA). The \textit{stellar mass complete sample} is constructed using the set of criteria detailed in Table \ref{table:sample_selection}.


\begin{deluxetable*}{cc}
  \tablenum{1}
  \tablecaption{The selection criteria used in \citet{2017MNRAS.472.2263H} to create the \textit{stellar mass-complete sample} of 6222 spiral galaxies.\label{table:sample_selection}}
  \tablewidth{0pt}
  \tablehead{
    \colhead{Description} & \colhead{Value}
  }
  \startdata
    Face-on spiral morphological selection. & GZ2 $p_\text{features} \times p_\text{not edge on} \times p_\text{spiral} \ge 0.5$ \\
    Redshift limits. & $0.02 < z < 0.055$ \\
    Face-on galaxy selection using $g$-band axial ratio. & $(b/a)_g > 0.4$ \\
    Mass limits for rough volume limited sample. & $9.45 < \log(M_* / M_\odot) \le 11.05$ \\
    Mass limits for complete sample\tablenotemark{a} & $2.07\log(z) + 12.64 < \log({M_* / M_\odot}) < 2.45\log(z) + 14.05$ \\
  \enddata
  \tablenotetext{a}{Stellar masses from \citet{2014ApJS..210....3M}}
  \tablenotetext{a}{Using the method of \citet{Pozzetti2009:0907.5416v2} and limits calculated by \citet{2017MNRAS.472.2263H}}
\end{deluxetable*}

The \textit{stellar mass-complete sample} was split into smaller sub-samples, each containing 100 galaxies. In an iterative process, each sub-sample was chosen to contain the 60 lowest redshift unclassified galaxies, and 40 random unclassified galaxies. This was done to ensure we would have an early sample to work with given the {\it a priori} unknown rate at which volunteers would provide classifications. Due to time constraints, classifications were only collected for two unique sub-samples. The mass-redshift relation of galaxies in the \textit{stellar mass-complete sample} from \citet{2017MNRAS.472.2263H} can be seen in Figure \ref{fig:mass_redshift}, with galaxies present in this work highlighted in red. Stellar Masses were calculated by \citet{2014ApJS..210....3M}.

\begin{figure*}
  \plotone{images__method/stellar_mass_selection_plot.pdf}
  \caption{Redshift against total galaxy stellar mass for all galaxies in the \textit{stellar mass-complete sample}, with the 198 galaxies considered in this paper highlighted in red. The distribution of stellar masses is shown in the right panel for the total sample and for the galaxies considered here. It is evident that the galaxies for which we collected classifications are not complete in stellar mass, but it is possible to select a further subset which would be.}
  \label{fig:mass_redshift}
\end{figure*}

In the first two sets of 100 galaxies, 1\% of galaxies (i.e. 2 images) failed to run through the image preparation process, due to an error when attempting to montage multiple frames. The root cause of this error is unknown, but it leaves a sample of 198 galaxies with images (the \textit{test sample}, 98 of which are repeated in a validation subset) that are considered in this paper, in order to explain the method used and test the reliability of the models obtained.

\subsubsection{Image and modelling metadata extraction}
\label{sec:image_creation}

The galaxy data shown to volunteers in the \textit{Galaxy Builder} project came in two forms: A gray-scale image cutout of the galaxy and a JSON file containing rendering information for the web-interface.

Both forms of data were obtained using a similar process:

\begin{enumerate}
\item A montage of multiple $r$-band corrected frames from the SDSS DR13 \citep{2017ApJS..233...25A} data release was created. To combine multiple FITS images, we made use of Astropy \citep{2018AJ....156..123A}, and the \textsc{Montage} \citep{2010arXiv1005.4454J} software package.
\item This montage was cropped to four times the Petrosian radius of the galaxy.
\item The \textsc{SExtractor} software \citep{source-extractor} was used to identify regions containing secondary sources (foreground stats, other galaxies) and generate a mask.
\item A PSF was obtained from the relevant Sloan $r$-band \texttt{psField} file, extracted at the central position of the galaxy \citep{2002AJ....123..485S}.
\item The JSON file was written containing the cut-out data and the 2D boolean mask obtained from the source extraction process. This file also contained other metadata needed for the rendering process (PSF, the size of the PSF array, and the width and height of the image).
\item Another JSON file containing simply the information used to render the volunteer's model (image size and PSF) was created.
\item An arcsinh-stretch was applied to the masked cutout (as described by \citealt{Lupton2003:astro-ph/0312483v1}). It was then saved as a grey-scale image.
\end{enumerate}

The decision to use $r$-band images in our subject set was due to its higher signal-to-noise than other bands.

Once a sub-sample had been created, the Zooniverse's \textsc{panoptes-python-client}\footnote{\url{https://github.com/zooniverse/panoptes-python-client}} was used to upload them as a subject-set to the Zooniverse.

The reprojection performed by \textsc{Montage} has a smoothing effect on the data, and thus does not conserve errors. We therefore create a separate stacked image, sigma image and corresponding pixel mask, using the same $r$-band corrected frames present in the montage. These images were not shown to volunteers but were used for model fitting and comparison.

\subsection{Choice of Retirement limit}
\label{sec:retirement-limit}

The number of independent answers needed to create reliable and reproducible aggregate classifications was not known at the start of this project. An initial experiment with collecting 10 classifications per galaxy demonstrated that this was insufficient; further experimentation with a diverse range of galaxy types (most with prominent spiral features including grand-design and flocculent arms) revealed 30 classifications per galaxy was sufficient.

The entire {\it test sample} of 198 galaxies was then presented to users, with 30 classifications collected per galaxy. In addition, one of the subsets was presented a second time, thus providing a validation subset to measure consistency between sets of 30 classifications on the same galaxies.

We also created 9 synthetic images of galaxies, containing various combinations of components available to volunteers and a spread of possible parameters. These synthetic galaxies were based off of a set of target galaxies from \textit{Galaxy Builder} and designed to be as realistic as possible, including the addition of realistic noise and pixel masks. This set of synthetic images was used to calibrate our aggregation and fitting methodology and thus is referred to as the \textit{calibration subset}.

\subsection{The Galaxy Model}
\label{section:galaxy-model}

Our chosen galaxy model was largely based on components described in \citet{galfit-paper}. The modelling code ignores masked regions identified as secondary sources by \textsc{SExtractor}. It over-samples the bulge, disc and bar components by a factor of five and performs PSF convolution using a PSF obtained from the relevant Sloan $r$-band \texttt{psField} file, extracted at the central position of the galaxy \citep{2002AJ....123..485S}. The model created by a volunteer could be chosen from

\begin{enumerate}
\item One exponential, ellipsoidal disc.
\item One ellipsoidal S\'ersic bulge, with $n$ chosen by volunteers.
\item One S\'ersic bar with a ``boxiness'' modifier (as described in \citealt{galfit-paper}), with $n$ and $c$ chosen by volunteers.
\item Any number of freehand poly-line\footnote{a poly-line, or polygonal chain, is a series of connected line segments.} spiral arms, as described below.
\end{enumerate}

\subsubsection{Spiral Arm Model}
Each spiral arm is modelled using a poly-line drawn by the volunteer. The brightness of a spiral arm at any point is given by the value of a Gaussian centred at the nearest point on any drawn poly-line, with volunteers able to choose the Gaussian width and peak brightness using sliders. Radial falloff was added by multiplying by the value of the previously added exponential disc, though volunteers could change the half-light radius of this falloff disc.

\subsection{Classification Aggregation Methodology}

In this Section, we will use the galaxy UGC 4721, a two-armed barred spiral galaxy at $z=0.02086$ classified by \citet{deVaucouleurs1991} as SBcd, to illustrate the data reduction and aggregation methodology. For UGC 4721 we received 32 classifications, containing 28 discs, 24 bulges, 17 bars and 47 drawn spiral arm poly-lines (four classifications did not contain spirals, seven contained one spiral arm, fourteen contained two arms, six contained three arms and one contained four arms). These annotations can be seen in Figure \ref{fig:drawn_shapes}, overlaid on the greyscale $r$-band image of the galaxy.

\begin{figure*}
  \plotone{images__method/drawn_shapes.pdf}
  \caption{Components drawn by volunteers for UGC 4721. The top left panel shows drawn discs, top right shows drawn bulges, bottom left shows drawn bars and bottom right shows drawn spiral arms. Discs, bulges and bars are displayed at twice their effective radii. These raw marks are subsequently aggregated to produce a consensus value for each galaxy component.}
  \label{fig:drawn_shapes}
\end{figure*}


\subsubsection{Aggregation of Volunteer Models}
\label{sec:aggregation_of_volunteer_models}

Aggregate model calculation was done on a component-by-component basis, rather than per classification, i.e. clustering of discs was performed independently to that of bulges, bars and spirals. We did not take into account any slider values, only the shape drawn by the volunteers. Disk classifications were doubled in effective radius to correct for a systematic error in disk size observed in the classifications received for the \textit{calibration subset}. Models parameters were clipped to be within the limits shown in Table \ref{table:bad_values} (deemed to be the physically acceptable bounds). All components were transformed from the coordinate space of the \textsc{Montage}-created images to the more accurate stacked images created for fitting. Clustering was performed using the Jaccard distance measure (also known as the intersect-over-union distance, or IOU distance), which is a simple metric determining the relative shared area of two sets:

\begin{equation}
d_J(A, B) = 1 - \frac{|A \cap B|}{|A \cup B|}.
\end{equation}

The algorithm chosen to perform clustering was the density-based spatial clustering of applications with noise (DBSCAN, \citealt{dbscan}) algorithm, due to its robustness and speed. We made use of Scikit-learn \citep{scikit-learn} to implement the algorithm. In DBSCAN the core of a cluster is defined as a group of at least \texttt{min\_points} that are all within a distance \texttt{eps} of each other. Additionally, any points within a distance \texttt{eps} of a cluster's core are also associated with the cluster.

\subsubsection{Disc, Bulge and Bar Clustering}

We select the disc clustering hyperparameters such that a disc is clustered for all galaxies, and the bulge hyperparameters to most successfully recover the morphology of galaxies in the \textit{calibration subset}. The value of \texttt{eps} used to cluster bars was tuned such that the aggregate model best agreed with GZ2 $p_\mathrm{bar}$ ($p_\mathrm{bar} < 0.2$ implying no bar and $p_\mathrm{bar} > 0.5$ implying a definite bar). The values chosen for \texttt{eps} were 0.3, 0.4, 0.478 for the disc, bulge and bar; \texttt{min\_points} was set to 4 for all three of these components.

We define the aggregate component to be the shape which minimises the sum of Jaccard distances to each of the members of the cluster. For our example galaxy, UGC 4721, clustered and aggregate components can be seen in Figure \ref{fig:mean_shapes}.

\subsubsection{Spiral Arm Clustering}
\label{sec:spiral_clustering}
To cluster drawn spiral arms, we define a custom separation measure to represent how far away one poly-line is from another. This measure was chosen to be the mean of the squared distances from each vertex in a poly-line to the nearest point (vertex or edge) of another poly-line, added to the mean of the squared distances from the second poly-line to the first. We make use of this separation measure inside the DBSCAN algorithm to cluster these drawn lines, after removing any self-intersecting drawn arms (as this was deemed an easy method to filter out ``bad'' classifications). Values of 0.001 and 4 were used for the \texttt{eps} and \texttt{min\_samples} hyper-parameters respectively.

Once spiral classifications on a galaxy have been clustered into the physical arms they represent, the points are deprojected using the axis ratio and position angle of the aggregated disc. The deprojection method assumes a thin disc and stretches the ellipsoidal minor axis to match the major axis.

Deprojected points within each drawn poly-line are converted to polar coordinates and unwound to allow model fitting. These unwound points are then cleaned using the Local-outlier-factor algorithm (LOF, \citealt{local-outlier-factor}). For each drawn poly-line in the cluster, the LOF algorithm was trained on all points not in that arm, and then used to predict whether each point in the arm should be considered an outlier. In this way we clean our data while respecting its grouped nature. The points removed as outliers for the example galaxy are shown in the bottom right panel of Figure \ref{fig:mean_shapes}.

\begin{figure*}
  \plotone{images__method/mean_shapes.pdf}
  \caption{Calculated aggregate components for UGC 4721. The aggregate disc is shown using a dot-dashed line and blue fill in the upper left panel, the aggregate bulge with a dotted line and orange fill in the upper right panel, the aggregate bar using a dashed line and green fill in the lower left panel and the aggregate spiral arms are plotted as red lines in the lower right panel. S\'ersic components are displayed at twice their effective radii. Black crosses in the lower right panel indicate spiral arm points that were identified as outliers and removed during cleaning (described in Section \ref{sec:spiral_clustering}). The aggregated components agree well with the underlying morphology, despite the noisiness of the classifications received.}
  \label{fig:mean_shapes}
\end{figure*}


For each arm cluster in each galaxy, a logarithmic spiral model was fitted using Bayesian Ridge Regression, performed using the Scikit-learn python package. Hyperpriors on the noise parameter were chosen by fitting a truncated gamma distribution \citep{2014arXiv1401.0287Z} to the spiral width slider values returned by volunteers (ignoring sliders left at the default or moved to the extremes of allowed values). Any logarithmic spirals within a distance of 0.0005 (given by the clustering metric) were deemed to be from the same arm and thus their classifications were merged and a log-spiral recalculated.

We do not assume that every arm in a galaxy has the same pitch angle. To obtain a single value for the pitch angle of a galaxy, we take the length-weighted average pitch angle of all arms detected in the galaxy (as used by \citealt{Davis2014:1402.1910v1}).

The galaxy model for UGC 4721 obtained through aggregation can be seen in the bottom left panel of Figure \ref{fig:model_tuning}.


\subsection{Error Estimation of Aggregate models}
\label{sec:error_estimation}

As all components in a cluster can be viewed as volunteers' attempts at modelling the true underlying component, the sample variance of the parameters of these shapes can be used as a measure of confidence in the parameters present in the aggregate result. These are highly sensitive to clustering hyperparameters, and are only valid for a component's position, size and shape. Figure \ref{fig:mean_shapes} illustrates the variance in clustered shapes for our example galaxy (UCG 4721); we see a large variation in the clustered discs, and much closer agreement on bulge and bar size and shape.

\subsection{Model Fitting}


The final step in creating \textit{Galaxy Builder} models is a numerical fit to fine-tune parameters. This fitting was performed using the L-BFGS-b algorithm \citep{doi:10.1137/0916069}, implemented in \textsc{Scipy} \citep{scipy-paper}. We minimize a custom likelihood function that assumes gaussian error on pixel values and incorporates the priors on parameters we obtain from clustering. The full fitting model and likelihood is detailed in Appendix \ref{sec:appendix_model_fitting}. We use the same model as used by volunteers in the online interface (with altered limits), with spiral arms restricted to being logarithmic spirals relative to the disc, and without the ability to change the relative falloff of spiral arms.

The model rendering and fitting code was written up using Google's JAX package \citep{jax2018github}, which allows GPU-optimization and automatic gradient calculation, enabling quick and accurate calculation of the jacobian matrix needed for the L-BFGS-b minimization algorithm.

We initially fit only for the brightnesses of components, and then simultaneously for all free parameters of all components. The result of the fit, including the final photometric model for UGC 4721, can be seen in \ref{fig:model_tuning}. The secondary components have been accounted for well, and the model has a sensible reduced chi-squared value of 1.176, where we have approximated degrees of freedom as the number of unmasked pixels present in the galaxy image (similar to \textsc{Galfit}).

\begin{figure*}
  \plotone{images__method/model_tuning.pdf}
  \caption{Effect of fitting on the aggregated models. The top left panel shows an Arcsinh-scaled image of the galaxy being fit (UGC 4721), the top middle shows the final model obtained (with the same limits and scaling as the galaxy image) and the top right shows the difference between the two images, in units of pixel uncertainty. The bottom panels show a simple representation of the model before and after tuning, overlaid on the galaxy image from the top-left panel. With minimal change to the aggregated components, we recover a detailed model that matches the galaxy exceptionally well, as evident in the residuals.}
  \label{fig:model_tuning}
\end{figure*}

We use the errors described in Section \ref{sec:error_estimation} as parameter uncertainties, as we feel an approach based on the local curvature of the likelihood-space (as used by \texttt{Galfit}) would likely fall foul of the issues described in the introduction and thus be a under-estimate. This decision means we do not have uncertainties for some parameters.

We remove two models for which a fit did not converge.
\end{document}
